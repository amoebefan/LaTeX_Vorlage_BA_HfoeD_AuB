%%%%%%%%%%%%%%%%%%%%%%%%%%%%%%%%%%%%%%%%%%%%%%%%%%%%%%%%%%%%%%%%%%%%%%%
%
% Dokumentenpräambel
% Für weitere Informationen zu den einzelnen Paketen siehe ctan.org
%
%%%%%%%%%%%%%%%%%%%%%%%%%%%%%%%%%%%%%%%%%%%%%%%%%%%%%%%%%%%%%%%%%%%%%%%

\documentclass[				%Dokumentenklassendefinition
	fontsize=11pt,			%Schriftgröße 
	paper=a4,			%Seitenformat
	bibliography=totoc, 		%Literaturverzeichnis im Inhaltsverzeichnis anzeigen
	listof=totoc, 			%(Tabellen- &) Abbildungsverzeichnis im Inhaltsverzeichnis anzeigen
	parskip=half,			%Abstand zwischen Absätzen
	]{scrreprt}						

\usepackage[utf8]{inputenc}			
\usepackage[T1]{fontenc}
\usepackage[ngerman]{babel}		%Spracheinstellung für neue deutsche Rechtschreibung
\usepackage{geometry}				
\usepackage{newtxtext} 			%eine an Times New Roman angelehnte Schriftart
\usepackage{graphicx, subfig}		%erweiterte Möglichkeiten Bilder einzufügen und zu bearbeiten
\usepackage{chngcntr}			%Zähler manipulieren
\counterwithout{footnote}{chapter}
\usepackage[onehalfspacing]{setspace}   %1,5-facher Zeilenabstand
\usepackage{nameref} 			%um mit \nameref Kapiteltitel zu referenzieren
\usepackage[hyphens]{url} 		%explizites Laden vom Paket url (!vor biblatex und href!) um die Option zur Trennung an Bindestrichen zu setzen (alternatives Paket breakurl)
\usepackage[german=guillemets]{csquotes}%Paket für Zitate mit Option zur Nutzung von Guillemets im Deutschen

\usepackage[	backend=biber, 		%in Optionen -> Texmaker konfigurieren Bib(la)tex auf "biber %"
			style=authoryear-icomp,
			sorting=nyt,
			isbn=false, 			%keine ISBN im Literaturverzeichnis
			ibidpage=true, 			%ebd. auf gleicher Seite
			ibidtracker=constrict, 	        %nur wenn in vorhergender Fußnote erwähnt
			maxbibnames = 99,		%Anzeige von bis zu 99 Autorennamen im Literaturverzeichnis
			]{biblatex}
			
\addbibresource{Literatur.bib}		%Einlesen der Datei mit den Literaturdaten		
			
%%%%%%%%%%%%%%%%%%%%%%%%%%%%%%%%%%%%%%%%%%%%%%%%%%%%%%%%%%%%%%%%%%%%%%%
% Individualisierung der Literaturangaben
%%%%%%%%%%%%%%%%%%%%%%%%%%%%%%%%%%%%%%%%%%%%%%%%%%%%%%%%%%%%%%%%%%%%%%%

\DefineBibliographyStrings{ngerman}{	%Text bei URLs ändern (Standard: besucht am)
  urlseen = {zuletzt aufgerufen am}
}

\setlength\bibnamesep{1.5\itemsep}	%Länge zwischen zwei Einträgen verschiedener Autoren
	% genauso möglich für
	% \bibitemsep für Abstand zwischen zwei verschiedenen Einträgen
	% \bibinitsep für Abstand zwischen zwei Einträgen von Autoren mit unterschiedlichem Anfangsbuchstaben

\DefineBibliographyStrings{german}%	%Anzeige von "et al." statt "u.a." bei vielen Autor:innen
	{andothers={et\addabbrvspace al\adddot}}

\urlstyle{same}				%URLs werden in der gleiche Schrift wie der Rest ausgegeben



%%%%%%%%%%%%%%%%%%%%%%%%%%%%%%%%%%%%%%%%%%%%%%%%%%%%%%%%%%%%%%%%%%%%%%%
% Weitere individuelle Anpassungen
%%%%%%%%%%%%%%%%%%%%%%%%%%%%%%%%%%%%%%%%%%%%%%%%%%%%%%%%%%%%%%%%%%%%%%%

%\usepackage{showframe}			%Anzeige des Seitenlayouts zur Überprüfung der Einstellungen
%\usepackage{blindtext}			%Zur Verwendung von "Dummy-Text" mit \blindtext, \Blindtext oder \Blinddocument

% Seitenzahlen rechts unten
\usepackage[autooneside=true]{scrlayer-scrpage}
\clearpairofpagestyles
\ofoot*{\pagemark}

%Fußnoten formatieren:
\deffootnote{2em}{1em}{% [Markenbreite (optional)]{Einzug}{Absatzeinzug}{Markendefinition 
			\makebox[2em][l]{\thefootnotemark}} 

% Abstand über Kapitelüberschrift ändern (https://www.texwelt.de/fragen/10230/wie-andere-ich-die-abstande-uber-und-unter-chaptertiteln)
\RedeclareSectionCommand[
  beforeskip=0pt,
  afterskip=1\baselineskip plus .1\baselineskip minus .167\baselineskip
]{chapter}

\setkomafont{disposition}{\normalcolor\bfseries} % Überschriften mit Standardschrift


% Tiefe der Nummerierung der Abschnitte und im Toc
\setcounter{secnumdepth}{3}		%Nummerierungstiefe der Abschnitte
\setcounter{tocdepth}{3}		%Auswertungstiefe des Inhaltsverzeichnisses


\usepackage[%				%PDF-Metadaten für Titel und Autor:in
	pdftitle={Titel der Bachelorarbeit}, 
	pdfauthor={Verfasser:in}]
	{hyperref}			%hyperref Paket sollte zur Kompatibilität möglichst als letztes geladen werden





%%%%%%%%%%%%%%%%%%%%%%%%%%%%%%%%%%%%%%%%%%%%%%%%%%%%%%%%%%%%%%%%%%%%%%%
% Dokumentenkörper
%%%%%%%%%%%%%%%%%%%%%%%%%%%%%%%%%%%%%%%%%%%%%%%%%%%%%%%%%%%%%%%%%%%%%%%

\begin{document}
\begin{titlepage}
	\singlespacing
	Hochschule für den öffentlichen Dienst in Bayern \\
	Fachbereich Archiv- und Bibliothekswesen \\
	Fachrichtung Bibliothekswesen \\
	\vspace{2cm}
	\begin{center}
		\textbf{Bachelorarbeit} \\
	    \vspace{2cm}
	    \textbf{Titel der Bachelorarbeit}
	\end{center}
	\vfill
	\begin{spacing}{2}
	Gutachter: \\
	Gutachter 1 \\
	Gutachter 2 \\
	\end{spacing}
	
	\vspace{2cm}
	
	\begin{minipage}[h]{7cm}
	Vorgelegt von: \\
	
	\end{minipage}
	\begin{minipage}[h]{7cm}
	Eingangsvermerk: \\
	
	\end{minipage}
\end{titlepage}
\newgeometry{left=2.5cm,right=3.5cm,top=2.5cm,bottom=2.5cm}
\pagenumbering{gobble}			%Ohne Seitenzahlen
%\include{Abstract} \newpage
\tableofcontents \newpage		%Inhaltsverzeicnis
\cleardoubleoddpage
\pagenumbering{arabic}			%Seitenzahlen mit arabischen Ziffern
%\include{Hauptteil}\newpage
\printbibliography \newpage 		%Literaturverzeichnis
%\listoffigures \newpage		%Abbildungsverzeichnis
\pagenumbering{Roman}			%Seitenzahlen mit großen römischen Ziffern
\chapter*{Rechtsverbindliche Erklärung}
\vspace{1cm}
Hiermit erkläre ich, dass ich die vorliegende Arbeit selbstständig und mit keinen anderen als den angeführten Hilfsmitteln erstellt habe. Die Erklärung schließt auch elektronische Dokumente ein. \vspace{2cm}

Ort, Datum \\

Unterschrift \newpage
\chapter*{Vereinbarung zur Benutzung der Bachelorarbeit}
\vspace{1cm}
Name, Vorname: \\

Titel der Bachelorarbeit: \\

\vspace{2cm}

\textit{Datenschutz/Personenbezogene Daten}:
\begin{enumerate}
\item Ich stimme zu / ich stimme nicht zu, dass der Titel meiner Bachelorarbeit zugleich mit meinem Autorennamen vom Fachbereich veröffentlicht werden kann, insbesondere im Internet öffentlich zugänglich gemacht werden kann. 
\end{enumerate}

\textit{Nutzungsrechte}:
\begin{enumerate}
\item[2.] Ich stimme zu / ich stimme nicht zu, dass ein Druckexemplar meiner Bachelorarbeit vom Fachbereich zur Benutzung an Dritte zur Verfügung gestellt werden kann.
\item[3.] Ich stimme auch zu / nicht zu, dass ein Druckexemplar meiner Bachelorarbeit der Bayerischen Staatsbibliothek übergeben wird, diese weist die Bachelorarbeit im öffentlich zugänglichen Bibliothekskatalog nach und stellt sie den Benutzern nach Maßgabe der Allgemeinen Benützungsordnung der Bayerischen Staatlichen Bibliotheken zur Verfügung.
\end{enumerate}

Hinweis: Soweit Sie selbst Ihre Bachelorarbeit veröffentlichen möchten, weisen wir Sie auf die Einhaltung der geltenden persönlichkeitsschutzrechtlichen, insbesondere datenschutzrechtlichen Vorschriften sowie die Vorschriften der Amtsverschwiegenheit hin. \vspace{2cm}

Ort, Datum \\

Unterschrift


\end{document}
